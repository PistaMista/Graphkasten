\documentclass{article}
\usepackage[T1]{fontenc}
\usepackage[unicode]{hyperref}
\usepackage{tikzuml/tikz-uml}

\title{Zettelkasten GTK}
\author{Kryštof Albrecht \\ \href{mailto:krystofalbrechtus@gmail.com}{krystofalbrechtus@gmail.com}}

\begin{document}
\maketitle
\tableofcontents

\newpage

\section{Introduction}

\subsection{Motivation}

Tools like Vimwiki, Vimzettel and Taskwiki provide a solid base for a Zettelkasten workflow. These are however based on \emph{the Vim text editor}, which has a lot of advantages:

\begin{itemize}
	\item Text editing cannot be faster
	
	\item Adding new notes is effortless

	\item The system is simple

	\item Tasks can have note links in them

	\item Any missing functionality can be added through Vimscript
\end{itemize}

The problem is however, that there is no GUI. Tools like \emph{Obsidian} offer a \emph{graph view} that enables natural viewing of the \emph{networked structure} of the Zettelkasten. This view allows faster and easier search and navigation, but also improves the chances of \emph{emergent ideas.}

This tool is based on a \emph{graph view} of the Zettelkasten and allows a wide range of filters to find precisely what we're looking for.

\subsection{Use cases}

The graph view is \emph{the basis} of the system and so will always be shown. The user will then be able to \emph{adjust} this graph view.

\begin{center}
\begin{tikzpicture}
	\umlactor{User}

	\begin{umlsystem}[x=4]{Zettelkasten GTK}
		\umlusecase[y=1]{Search notes}
		\umlusecase[x=4]{Show result in graph}

		\umlusecase[y=-1]{Show neighboring notes}

		\umlusecase[y=-3]{Show HTML preview}
		\umlusecase[y=-4,x=4]{Highlight hovered link}

		\umlusecase[y=-2]{Open note in Neovim}
	\end{umlsystem}

	\umlextend{usecase-2}{usecase-1}
	\umlinclude{usecase-5}{usecase-4}

	\umlassoc{User}{usecase-1}
	\umlassoc{User}{usecase-3}
	\umlassoc{User}{usecase-4}
	\umlassoc{User}{usecase-6}
\end{tikzpicture}
\end{center}

\subsection{Other features}

\end{document}
